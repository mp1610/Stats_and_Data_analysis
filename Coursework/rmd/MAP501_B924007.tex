% Options for packages loaded elsewhere
\PassOptionsToPackage{unicode}{hyperref}
\PassOptionsToPackage{hyphens}{url}
%
\documentclass[
]{article}
\usepackage{amsmath,amssymb}
\usepackage{lmodern}
\usepackage{iftex}
\ifPDFTeX
  \usepackage[T1]{fontenc}
  \usepackage[utf8]{inputenc}
  \usepackage{textcomp} % provide euro and other symbols
\else % if luatex or xetex
  \usepackage{unicode-math}
  \defaultfontfeatures{Scale=MatchLowercase}
  \defaultfontfeatures[\rmfamily]{Ligatures=TeX,Scale=1}
\fi
% Use upquote if available, for straight quotes in verbatim environments
\IfFileExists{upquote.sty}{\usepackage{upquote}}{}
\IfFileExists{microtype.sty}{% use microtype if available
  \usepackage[]{microtype}
  \UseMicrotypeSet[protrusion]{basicmath} % disable protrusion for tt fonts
}{}
\makeatletter
\@ifundefined{KOMAClassName}{% if non-KOMA class
  \IfFileExists{parskip.sty}{%
    \usepackage{parskip}
  }{% else
    \setlength{\parindent}{0pt}
    \setlength{\parskip}{6pt plus 2pt minus 1pt}}
}{% if KOMA class
  \KOMAoptions{parskip=half}}
\makeatother
\usepackage{xcolor}
\usepackage[margin=1in]{geometry}
\usepackage{color}
\usepackage{fancyvrb}
\newcommand{\VerbBar}{|}
\newcommand{\VERB}{\Verb[commandchars=\\\{\}]}
\DefineVerbatimEnvironment{Highlighting}{Verbatim}{commandchars=\\\{\}}
% Add ',fontsize=\small' for more characters per line
\usepackage{framed}
\definecolor{shadecolor}{RGB}{248,248,248}
\newenvironment{Shaded}{\begin{snugshade}}{\end{snugshade}}
\newcommand{\AlertTok}[1]{\textcolor[rgb]{0.94,0.16,0.16}{#1}}
\newcommand{\AnnotationTok}[1]{\textcolor[rgb]{0.56,0.35,0.01}{\textbf{\textit{#1}}}}
\newcommand{\AttributeTok}[1]{\textcolor[rgb]{0.77,0.63,0.00}{#1}}
\newcommand{\BaseNTok}[1]{\textcolor[rgb]{0.00,0.00,0.81}{#1}}
\newcommand{\BuiltInTok}[1]{#1}
\newcommand{\CharTok}[1]{\textcolor[rgb]{0.31,0.60,0.02}{#1}}
\newcommand{\CommentTok}[1]{\textcolor[rgb]{0.56,0.35,0.01}{\textit{#1}}}
\newcommand{\CommentVarTok}[1]{\textcolor[rgb]{0.56,0.35,0.01}{\textbf{\textit{#1}}}}
\newcommand{\ConstantTok}[1]{\textcolor[rgb]{0.00,0.00,0.00}{#1}}
\newcommand{\ControlFlowTok}[1]{\textcolor[rgb]{0.13,0.29,0.53}{\textbf{#1}}}
\newcommand{\DataTypeTok}[1]{\textcolor[rgb]{0.13,0.29,0.53}{#1}}
\newcommand{\DecValTok}[1]{\textcolor[rgb]{0.00,0.00,0.81}{#1}}
\newcommand{\DocumentationTok}[1]{\textcolor[rgb]{0.56,0.35,0.01}{\textbf{\textit{#1}}}}
\newcommand{\ErrorTok}[1]{\textcolor[rgb]{0.64,0.00,0.00}{\textbf{#1}}}
\newcommand{\ExtensionTok}[1]{#1}
\newcommand{\FloatTok}[1]{\textcolor[rgb]{0.00,0.00,0.81}{#1}}
\newcommand{\FunctionTok}[1]{\textcolor[rgb]{0.00,0.00,0.00}{#1}}
\newcommand{\ImportTok}[1]{#1}
\newcommand{\InformationTok}[1]{\textcolor[rgb]{0.56,0.35,0.01}{\textbf{\textit{#1}}}}
\newcommand{\KeywordTok}[1]{\textcolor[rgb]{0.13,0.29,0.53}{\textbf{#1}}}
\newcommand{\NormalTok}[1]{#1}
\newcommand{\OperatorTok}[1]{\textcolor[rgb]{0.81,0.36,0.00}{\textbf{#1}}}
\newcommand{\OtherTok}[1]{\textcolor[rgb]{0.56,0.35,0.01}{#1}}
\newcommand{\PreprocessorTok}[1]{\textcolor[rgb]{0.56,0.35,0.01}{\textit{#1}}}
\newcommand{\RegionMarkerTok}[1]{#1}
\newcommand{\SpecialCharTok}[1]{\textcolor[rgb]{0.00,0.00,0.00}{#1}}
\newcommand{\SpecialStringTok}[1]{\textcolor[rgb]{0.31,0.60,0.02}{#1}}
\newcommand{\StringTok}[1]{\textcolor[rgb]{0.31,0.60,0.02}{#1}}
\newcommand{\VariableTok}[1]{\textcolor[rgb]{0.00,0.00,0.00}{#1}}
\newcommand{\VerbatimStringTok}[1]{\textcolor[rgb]{0.31,0.60,0.02}{#1}}
\newcommand{\WarningTok}[1]{\textcolor[rgb]{0.56,0.35,0.01}{\textbf{\textit{#1}}}}
\usepackage{graphicx}
\makeatletter
\def\maxwidth{\ifdim\Gin@nat@width>\linewidth\linewidth\else\Gin@nat@width\fi}
\def\maxheight{\ifdim\Gin@nat@height>\textheight\textheight\else\Gin@nat@height\fi}
\makeatother
% Scale images if necessary, so that they will not overflow the page
% margins by default, and it is still possible to overwrite the defaults
% using explicit options in \includegraphics[width, height, ...]{}
\setkeys{Gin}{width=\maxwidth,height=\maxheight,keepaspectratio}
% Set default figure placement to htbp
\makeatletter
\def\fps@figure{htbp}
\makeatother
\setlength{\emergencystretch}{3em} % prevent overfull lines
\providecommand{\tightlist}{%
  \setlength{\itemsep}{0pt}\setlength{\parskip}{0pt}}
\setcounter{secnumdepth}{-\maxdimen} % remove section numbering
\ifLuaTeX
  \usepackage{selnolig}  % disable illegal ligatures
\fi
\IfFileExists{bookmark.sty}{\usepackage{bookmark}}{\usepackage{hyperref}}
\IfFileExists{xurl.sty}{\usepackage{xurl}}{} % add URL line breaks if available
\urlstyle{same} % disable monospaced font for URLs
\hypersetup{
  pdftitle={MAP501 Coursework Submission},
  hidelinks,
  pdfcreator={LaTeX via pandoc}}

\title{MAP501 Coursework Submission}
\author{}
\date{\vspace{-2.5em}}

\begin{document}
\maketitle

{
\setcounter{tocdepth}{2}
\tableofcontents
}
\hypertarget{preamble}{%
\section{Preamble}\label{preamble}}

\begin{Shaded}
\begin{Highlighting}[]
\FunctionTok{library}\NormalTok{(}\StringTok{"rio"}\NormalTok{)}
\FunctionTok{library}\NormalTok{(}\StringTok{"dplyr"}\NormalTok{)}
\FunctionTok{library}\NormalTok{(}\StringTok{"tidyr"}\NormalTok{)}
\FunctionTok{library}\NormalTok{(}\StringTok{"magrittr"}\NormalTok{)}
\FunctionTok{library}\NormalTok{(}\StringTok{"ggplot2"}\NormalTok{)}
\FunctionTok{library}\NormalTok{(}\StringTok{"pROC"}\NormalTok{)}
\FunctionTok{library}\NormalTok{(}\StringTok{"car"}\NormalTok{)}
\FunctionTok{library}\NormalTok{(}\StringTok{"nnet"}\NormalTok{)}
\FunctionTok{library}\NormalTok{(}\StringTok{"caret"}\NormalTok{)}
\FunctionTok{library}\NormalTok{(}\StringTok{"lme4"}\NormalTok{)}
\FunctionTok{library}\NormalTok{(}\StringTok{"AmesHousing"}\NormalTok{)}
\FunctionTok{library}\NormalTok{(}\StringTok{"here"}\NormalTok{) }
\FunctionTok{library}\NormalTok{(}\StringTok{"tidyverse"}\NormalTok{) }
\FunctionTok{library}\NormalTok{(}\StringTok{"lubridate"}\NormalTok{)}
\FunctionTok{library}\NormalTok{(}\StringTok{"janitor"}\NormalTok{)}
\FunctionTok{library}\NormalTok{(}\StringTok{"ggrepel"}\NormalTok{)}
\FunctionTok{library}\NormalTok{(}\StringTok{"sandwich"}\NormalTok{)}
\FunctionTok{library}\NormalTok{(}\StringTok{"investr"}\NormalTok{)}
\FunctionTok{library}\NormalTok{(}\StringTok{"rcompanion"}\NormalTok{)}
\FunctionTok{library}\NormalTok{(}\StringTok{"ggcorrplot"}\NormalTok{)}
\FunctionTok{library}\NormalTok{(}\StringTok{"corrr"}\NormalTok{)}
\FunctionTok{library}\NormalTok{(}\StringTok{"effects"}\NormalTok{)}

\NormalTok{Ames }\OtherTok{\textless{}{-}} \FunctionTok{make\_ames}\NormalTok{()}
\end{Highlighting}
\end{Shaded}

\hypertarget{data-preparation}{%
\section{1. Data Preparation}\label{data-preparation}}

\hypertarget{a}{%
\subsection{1a)}\label{a}}

Import the soccer.csv dataset as ``footballer\_data''. (2 points)

\begin{Shaded}
\begin{Highlighting}[]
\NormalTok{footballer\_data }\OtherTok{\textless{}{-}} 
  \FunctionTok{read\_csv}\NormalTok{(}\FunctionTok{here}\NormalTok{(}\StringTok{"data/soccer.csv"}\NormalTok{))}
\end{Highlighting}
\end{Shaded}

\hypertarget{b}{%
\subsection{1b)}\label{b}}

Ensure all character variables are treated as factors and where variable
names have a space, rename the variables without these. (3 points)

\begin{Shaded}
\begin{Highlighting}[]
\NormalTok{footballer\_data }\OtherTok{\textless{}{-}}
\NormalTok{  footballer\_data }\SpecialCharTok{\%\textgreater{}\%}
  \FunctionTok{clean\_names}\NormalTok{() }\SpecialCharTok{\%\textgreater{}\%}
  \FunctionTok{mutate\_at}\NormalTok{(}\FunctionTok{vars}\NormalTok{(full\_name, birthday\_gmt, position, current\_club, nationality),}
            \FunctionTok{list}\NormalTok{(factor))}

\NormalTok{footballer\_data}
\end{Highlighting}
\end{Shaded}

\begin{verbatim}
# A tibble: 570 x 45
   full_~1   age birth~2 birth~3 posit~4 curre~5 minut~6 minut~7 minut~8 natio~9
   <fct>   <dbl>   <dbl> <fct>   <fct>   <fct>     <dbl>   <dbl>   <dbl> <fct>  
 1 Aaron ~    32  6.30e8 15/12/~ Defend~ West H~    1589     888     701 England
 2 Aaron ~    35  5.46e8 16/04/~ Midfie~ Burnley    1217     487     730 England
 3 Aaron ~    31  6.53e8 15/09/~ Midfie~ Hudder~    2327    1190    1137 Austra~
 4 Aaron ~    31  6.62e8 26/12/~ Midfie~ Arsenal    1327     689     638 Wales  
 5 Aaron ~    22  9.68e8 07/09/~ Forward Hudder~      69      14      55 England
 6 Aaron ~    24  8.81e8 26/11/~ Midfie~ Crysta~    3135    1605    1530 England
 7 Abdelh~    25  8.49e8 28/11/~ Midfie~ Hudder~      49       0      49 Morocco
 8 Abdoul~    29  7.26e8 01/01/~ Midfie~ Watford    3062    1566    1496 France 
 9 Abouba~    27  7.95e8 07/03/~ Forward Fulham      687     468     219 France 
10 Adalbe~    25  8.65e8 31/05/~ Forward Watford       0       0       0 Venezu~
# ... with 560 more rows, 35 more variables: appearances_overall <dbl>,
#   appearances_home <dbl>, appearances_away <dbl>, goals_overall <dbl>,
#   goals_home <dbl>, goals_away <dbl>, assists_overall <dbl>,
#   assists_home <dbl>, assists_away <dbl>, penalty_goals <dbl>,
#   penalty_misses <dbl>, clean_sheets_overall <dbl>, clean_sheets_home <dbl>,
#   clean_sheets_away <dbl>, conceded_overall <dbl>, conceded_home <dbl>,
#   conceded_away <dbl>, yellow_cards_overall <dbl>, ...
\end{verbatim}

\hypertarget{c}{%
\subsection{1c)}\label{c}}

Remove the columns birthday and birthday\_GMT. (2 points)

\begin{Shaded}
\begin{Highlighting}[]
\NormalTok{footballer\_data2 }\OtherTok{\textless{}{-}}
\NormalTok{  footballer\_data }\SpecialCharTok{\%\textgreater{}\%}
  \FunctionTok{select}\NormalTok{(}\SpecialCharTok{{-}}\FunctionTok{c}\NormalTok{(birthday, birthday\_gmt))}

\NormalTok{footballer\_data2}
\end{Highlighting}
\end{Shaded}

\begin{verbatim}
# A tibble: 570 x 43
   full_~1   age posit~2 curre~3 minut~4 minut~5 minut~6 natio~7 appea~8 appea~9
   <fct>   <dbl> <fct>   <fct>     <dbl>   <dbl>   <dbl> <fct>     <dbl>   <dbl>
 1 Aaron ~    32 Defend~ West H~    1589     888     701 England      20      11
 2 Aaron ~    35 Midfie~ Burnley    1217     487     730 England      16       7
 3 Aaron ~    31 Midfie~ Hudder~    2327    1190    1137 Austra~      29      15
 4 Aaron ~    31 Midfie~ Arsenal    1327     689     638 Wales        28      14
 5 Aaron ~    22 Forward Hudder~      69      14      55 England       2       1
 6 Aaron ~    24 Midfie~ Crysta~    3135    1605    1530 England      35      18
 7 Abdelh~    25 Midfie~ Hudder~      49       0      49 Morocco       2       0
 8 Abdoul~    29 Midfie~ Watford    3062    1566    1496 France       35      18
 9 Abouba~    27 Forward Fulham      687     468     219 France       13       8
10 Adalbe~    25 Forward Watford       0       0       0 Venezu~       0       0
# ... with 560 more rows, 33 more variables: appearances_away <dbl>,
#   goals_overall <dbl>, goals_home <dbl>, goals_away <dbl>,
#   assists_overall <dbl>, assists_home <dbl>, assists_away <dbl>,
#   penalty_goals <dbl>, penalty_misses <dbl>, clean_sheets_overall <dbl>,
#   clean_sheets_home <dbl>, clean_sheets_away <dbl>, conceded_overall <dbl>,
#   conceded_home <dbl>, conceded_away <dbl>, yellow_cards_overall <dbl>,
#   red_cards_overall <dbl>, goals_involved_per_90_overall <dbl>, ...
\end{verbatim}

\hypertarget{d}{%
\subsection{1d)}\label{d}}

Remove the cases with age\textless=15 and age\textgreater40. (2 points)

\begin{Shaded}
\begin{Highlighting}[]
\NormalTok{footballer\_data3 }\OtherTok{\textless{}{-}}
\NormalTok{  footballer\_data2 }\SpecialCharTok{\%\textgreater{}\%}
  \FunctionTok{filter}\NormalTok{(age }\SpecialCharTok{\textgreater{}} \DecValTok{15} \SpecialCharTok{\&}\NormalTok{ age }\SpecialCharTok{\textless{}=} \DecValTok{40}\NormalTok{)}

\FunctionTok{max}\NormalTok{(footballer\_data3}\SpecialCharTok{$}\NormalTok{age)}
\FunctionTok{min}\NormalTok{(footballer\_data3}\SpecialCharTok{$}\NormalTok{age)}
\NormalTok{footballer\_data3}
\end{Highlighting}
\end{Shaded}

\begin{verbatim}
[1] 40
[1] 20
# A tibble: 565 x 43
   full_~1   age posit~2 curre~3 minut~4 minut~5 minut~6 natio~7 appea~8 appea~9
   <fct>   <dbl> <fct>   <fct>     <dbl>   <dbl>   <dbl> <fct>     <dbl>   <dbl>
 1 Aaron ~    32 Defend~ West H~    1589     888     701 England      20      11
 2 Aaron ~    35 Midfie~ Burnley    1217     487     730 England      16       7
 3 Aaron ~    31 Midfie~ Hudder~    2327    1190    1137 Austra~      29      15
 4 Aaron ~    31 Midfie~ Arsenal    1327     689     638 Wales        28      14
 5 Aaron ~    22 Forward Hudder~      69      14      55 England       2       1
 6 Aaron ~    24 Midfie~ Crysta~    3135    1605    1530 England      35      18
 7 Abdelh~    25 Midfie~ Hudder~      49       0      49 Morocco       2       0
 8 Abdoul~    29 Midfie~ Watford    3062    1566    1496 France       35      18
 9 Abouba~    27 Forward Fulham      687     468     219 France       13       8
10 Adalbe~    25 Forward Watford       0       0       0 Venezu~       0       0
# ... with 555 more rows, 33 more variables: appearances_away <dbl>,
#   goals_overall <dbl>, goals_home <dbl>, goals_away <dbl>,
#   assists_overall <dbl>, assists_home <dbl>, assists_away <dbl>,
#   penalty_goals <dbl>, penalty_misses <dbl>, clean_sheets_overall <dbl>,
#   clean_sheets_home <dbl>, clean_sheets_away <dbl>, conceded_overall <dbl>,
#   conceded_home <dbl>, conceded_away <dbl>, yellow_cards_overall <dbl>,
#   red_cards_overall <dbl>, goals_involved_per_90_overall <dbl>, ...
\end{verbatim}

\hypertarget{linear-regression}{%
\section{2. Linear Regression}\label{linear-regression}}

\hypertarget{a-1}{%
\subsection{2a)}\label{a-1}}

\hypertarget{bi}{%
\subsection{2bi)}\label{bi}}

\hypertarget{bii}{%
\subsection{2bii)}\label{bii}}

\hypertarget{biii}{%
\subsection{2biii)}\label{biii}}

\hypertarget{c-1}{%
\subsection{2c)}\label{c-1}}

\hypertarget{d-1}{%
\subsection{2d)}\label{d-1}}

\hypertarget{e}{%
\subsection{2e)}\label{e}}

\hypertarget{f}{%
\subsection{2f)}\label{f}}

\hypertarget{g}{%
\subsection{2g)}\label{g}}

\hypertarget{h}{%
\subsection{2h)}\label{h}}

\hypertarget{i}{%
\subsection{2i)}\label{i}}

\hypertarget{j}{%
\subsection{2j)}\label{j}}

\hypertarget{k}{%
\subsection{2k)}\label{k}}

\hypertarget{l}{%
\subsection{2l)}\label{l}}

\hypertarget{m}{%
\subsection{2m)}\label{m}}

\hypertarget{n}{%
\subsection{2n)}\label{n}}

\hypertarget{logistic-regression}{%
\section{3. Logistic Regression}\label{logistic-regression}}

\hypertarget{ai}{%
\subsection{3ai)}\label{ai}}

\hypertarget{aii}{%
\subsection{3aii)}\label{aii}}

\hypertarget{aiii}{%
\subsection{3aiii)}\label{aiii}}

\hypertarget{aiv}{%
\subsection{3aiv)}\label{aiv}}

\hypertarget{b-1}{%
\subsection{3b)}\label{b-1}}

\hypertarget{c-2}{%
\subsection{3c)}\label{c-2}}

\hypertarget{multinomial-regression}{%
\section{4. Multinomial Regression}\label{multinomial-regression}}

\hypertarget{a-2}{%
\subsection{4a)}\label{a-2}}

\hypertarget{b-2}{%
\subsection{4b)}\label{b-2}}

\hypertarget{c-3}{%
\subsection{4c)}\label{c-3}}

\hypertarget{poissonquasipoisson-regression}{%
\section{5. Poisson/quasipoisson
Regression}\label{poissonquasipoisson-regression}}

\hypertarget{a-do-i-need-dimensionality-missingness-and-correlations}{%
\subsection{5a) \# do i need dimensionality, missingness, and
correlations???}\label{a-do-i-need-dimensionality-missingness-and-correlations}}

For the ``footballer\_data'' dataset, create a model appearances\_mod to
predict the total number of overall appearances a player had based on
position and age. (2 points)

\begin{Shaded}
\begin{Highlighting}[]
\CommentTok{\# does it want a LaTex equation or??? do i need to use predict function???}
\NormalTok{appearances\_mod }\OtherTok{\textless{}{-}}
  \FunctionTok{glm}\NormalTok{(appearances\_overall}\SpecialCharTok{\textasciitilde{}}\NormalTok{position }\SpecialCharTok{+}\NormalTok{ age, }\AttributeTok{data =}\NormalTok{ footballer\_data3, }\AttributeTok{family =} \StringTok{"poisson"}\NormalTok{)}

\FunctionTok{summary}\NormalTok{(appearances\_mod)}
\end{Highlighting}
\end{Shaded}

\begin{verbatim}

Call:
glm(formula = appearances_overall ~ position + age, family = "poisson", 
    data = footballer_data3)

Deviance Residuals: 
    Min       1Q   Median       3Q      Max  
-7.5377  -3.5215   0.0351   2.1892   6.1853  

Coefficients:
                    Estimate Std. Error z value Pr(>|z|)    
(Intercept)         1.575316   0.074884  21.037  < 2e-16 ***
positionForward     0.110606   0.027448   4.030 5.59e-05 ***
positionGoalkeeper -0.364605   0.040780  -8.941  < 2e-16 ***
positionMidfielder  0.118259   0.023309   5.074 3.90e-07 ***
age                 0.043704   0.002392  18.275  < 2e-16 ***
---
Signif. codes:  0 '***' 0.001 '**' 0.01 '*' 0.05 '.' 0.1 ' ' 1

(Dispersion parameter for poisson family taken to be 1)

    Null deviance: 6539.7  on 564  degrees of freedom
Residual deviance: 6114.4  on 560  degrees of freedom
AIC: 8417.1

Number of Fisher Scoring iterations: 5
\end{verbatim}

\hypertarget{b-3}{%
\subsection{5b)}\label{b-3}}

Check the assumption of the model using a diagnostic plot and comment on
your findings. (3 points)

Using a poisson model assumes that mean = variance, hence the dispersion
parameter is assumed to be 1. This can be tested using a plot of
`Absolute value of residuals' versus `Predicted Means'. which should
look flat and hover around 0.8 (green line).

\begin{Shaded}
\begin{Highlighting}[]
\FunctionTok{plot}\NormalTok{(appearances\_mod,}\AttributeTok{which=}\DecValTok{3}\NormalTok{)}
\FunctionTok{abline}\NormalTok{(}\AttributeTok{h=}\FloatTok{0.8}\NormalTok{,}\AttributeTok{col=}\DecValTok{3}\NormalTok{)}
\end{Highlighting}
\end{Shaded}

\begin{center}\includegraphics{MAP501_B924007_files/figure-latex/unnamed-chunk-7-1} \end{center}

The figure above shows that the red line (data set result) is not
perfectly flat and does not hover around the green line (0.8), hence the
model is not the most reasonable to use for this question. The model
shows the data is heavily overdispersed, which could also suggest that
we have not accounted for all of the important predictors in our model.

Another assumption of the Poisson model is linearity. The figure below
shows the residuals vs fitted data, to see if the model data (red line)
fits in with the assumption of linearity. It is evident that this is not
the case as the red line slowly increases from approximately -2 until 1
(estimate), after which it decreases linearly to -3. This completely
goes against the assumption of linearity as the plot is not flat at all.

\begin{Shaded}
\begin{Highlighting}[]
\FunctionTok{plot}\NormalTok{(appearances\_mod, }\AttributeTok{which =} \DecValTok{1}\NormalTok{)}
\end{Highlighting}
\end{Shaded}

\begin{center}\includegraphics{MAP501_B924007_files/figure-latex/unnamed-chunk-8-1} \end{center}

The next assumption made for a Poisson Model is the assumption of
Distribution. For deviance residuals, we investigage a qqplot to see if
the data increases linearly. The figure below shows that the model is
not perfect as the values deviate from the qqplot line towards the start
and end of the x-axis (Theoretical Quantities). Between x-values of -1
to 1 however, the data satisfies the qqplot, suggesting that the Poisson
Model is not completely incorrect in satisfying the distribution
assumption.

\begin{Shaded}
\begin{Highlighting}[]
\FunctionTok{plot}\NormalTok{(appearances\_mod, }\AttributeTok{which =} \DecValTok{2}\NormalTok{)}
\end{Highlighting}
\end{Shaded}

\begin{center}\includegraphics{MAP501_B924007_files/figure-latex/unnamed-chunk-9-1} \end{center}

Thirdly, a Poisson Model assumes Independence. This investigates
residuals as a function of order of datapoints for evidence of
``snaking''. However, since the dataset is not of natural order, this
cannot be investigated further.

As the Poisson Model above does not satisfy most assumptions such as
`the linearity assumption' and `the variance = mean assumption', it is
worth creating a model where variance is not equal to mean, hence the
dispersion parameter is not equal to 1. This can be done using a
quasipoisson model which assumes that dispersion is a linear function of
mean. The summary below shows that the dispersion parameter is not equal
to 1, instead it is equal to 8.95. As the dispersion parameter is
greater than 1, it still suggests that there is overdispersion in the
data.

\begin{Shaded}
\begin{Highlighting}[]
\NormalTok{quasiappearances\_mod }\OtherTok{\textless{}{-}}
  \FunctionTok{glm}\NormalTok{(appearances\_overall}\SpecialCharTok{\textasciitilde{}}\NormalTok{position }\SpecialCharTok{+}\NormalTok{ age, }\AttributeTok{data =}\NormalTok{ footballer\_data3, }\AttributeTok{family =} \StringTok{"quasipoisson"}\NormalTok{)}

\FunctionTok{summary}\NormalTok{(quasiappearances\_mod)}
\end{Highlighting}
\end{Shaded}

\begin{verbatim}

Call:
glm(formula = appearances_overall ~ position + age, family = "quasipoisson", 
    data = footballer_data3)

Deviance Residuals: 
    Min       1Q   Median       3Q      Max  
-7.5377  -3.5215   0.0351   2.1892   6.1853  

Coefficients:
                    Estimate Std. Error t value Pr(>|t|)    
(Intercept)         1.575316   0.223980   7.033 5.90e-12 ***
positionForward     0.110606   0.082097   1.347  0.17844    
positionGoalkeeper -0.364605   0.121975  -2.989  0.00292 ** 
positionMidfielder  0.118259   0.069717   1.696  0.09039 .  
age                 0.043704   0.007153   6.110 1.87e-09 ***
---
Signif. codes:  0 '***' 0.001 '**' 0.01 '*' 0.05 '.' 0.1 ' ' 1

(Dispersion parameter for quasipoisson family taken to be 8.946343)

    Null deviance: 6539.7  on 564  degrees of freedom
Residual deviance: 6114.4  on 560  degrees of freedom
AIC: NA

Number of Fisher Scoring iterations: 5
\end{verbatim}

Using a plot of `Absolute value of residuals' versus `Predicted Means',
the quasipoisson model can be evaluated too assuming the model line (red
line) should be fairly linear and relatively close to the 0.8 (green
line). While the figure below is not perfect, it is definitely an
improvement from the Poisson Model in which the Residual value was not
even close to 0.8, nor was it as linear.

\begin{Shaded}
\begin{Highlighting}[]
\FunctionTok{plot}\NormalTok{(quasiappearances\_mod,}\AttributeTok{which=}\DecValTok{3}\NormalTok{)}
\FunctionTok{abline}\NormalTok{(}\AttributeTok{h=}\FloatTok{0.8}\NormalTok{,}\AttributeTok{col=}\DecValTok{3}\NormalTok{)}
\end{Highlighting}
\end{Shaded}

\begin{center}\includegraphics{MAP501_B924007_files/figure-latex/unnamed-chunk-11-1} \end{center}

\hypertarget{c-4}{%
\subsection{5c)}\label{c-4}}

What do the coefficients of the model tell us about? which position has
the most appearances? How many times more appearances do forwards get on
average than goalkeepers? (3 points)

\end{document}
